\documentclass[12pt]{article}
%\usepackage[
%backend=bibtex,
%citestyle=ieee
%]{biblatex}
%\addbibresource{RI2020.bib}
%\renewbibmacro{in:}{}
%% Define some new colors
\usepackage{xcolor}

\usepackage{scrextend}


%% Page and text formatting
\usepackage[left=1.0in, right=1.0in, top=1.0in, bottom=1.0in]{geometry} % margins
\usepackage{setspace}
\singlespacing % No more than 6 lines of text per inch
\usepackage{amsmath, amsfonts}
\usepackage[T1]{fontenc}
\usepackage{times}
\usepackage{graphics}
\usepackage{makecell}
\usepackage{tabularx}
\usepackage{indentfirst}

%% Set up hyperlinks
\usepackage[linktoc=page]{hyperref}
\hypersetup{
  colorlinks = true,
  urlcolor = blue,
  citecolor = blue,
  linkcolor = blue,
}

%% Reduce whitespace around lists (globally)
\usepackage{enumitem}
\setlist[itemize]{nosep,left=0pt}
\setlist[enumerate]{nosep}
\setlist[description]{nosep}

%% Formatting for sections

\usepackage{sectsty}
\sectionfont{\fontsize{22}{25}\selectfont}
\subsectionfont{\fontsize{15}{25}\selectfont}




\begin{document}

\fontsize{10}{12}\selectfont

\linespread{1.3}
\fontfamily{ptm}\selectfont

\vspace{2mm}

\vspace{1mm}

\noindent{\fontsize{15}{17}\selectfont Kevin Nash \\ }

\noindent\begin{tabular*}{\textwidth}{@{\extracolsep{\fill}}l l}
Research Associate & Email: \href{mailto:knash@physics.rutgers.edu}{knash@physics.rutgers.edu} \\
Rutgers University & Phone: (434) 760-1424\\
Department of Physics \& Astronomy \\
136 Frelinghuysen Rd\\
Piscataway, NJ 08854\\
\hline
\\
\\
\end{tabular*}




\section*{Diversity Statement}

One of the most important challenges facing the sciences is how to cultivate
diversity without lowering physics standards and still allowing equal opportunities.
Objectively, a diverse field leads to a larger community and greater public interest as well
as insights from personal experience.  Diversity is a complicated issue,
and unfortunately, one that is larger than the
Physics community. Cultural norms have been set on an unhealthy path for
generations, and are just recently starting to support equality.  So we
can either wait for society to self-correct, or we can be proactive.

In order to expand the interest for a career in the sciences, we need to
interact with people while they are still developing the idea of cultural norms.
From a faculty point of view, this means expanding the outreach
program. In my undergraduate and graduate careers, I have gained experience on how
to inspire people at these events.  There are an array of interesting and
fun demonstrations that can pique the interest of people at this pivotal point of
development. In my physics career, I have gained experience helping
with outreach, from the yearly university shows to the USA Science and Engineering Festival.
During these outreach opportunities, I would perform various demonstrations --
my favorite of which is the cloud chamber, which offers a priceless opportunity
to show particle physics in action.  However, still, these outreach opportunities
can be better directed. There are certainly ways of targeting groups that are
underrepresented in the physics community, for example, a careful
determination of the community location for the event or through
targeted advertisement.  I have experience with the effect that this type of
outreach can have from graduate school in Baltimore, where
we would travel to local schools and perform basic physics presentations. In
this way, we can reach out to schools directly that have a higher underrepresented
population and cultivate an interest in the field.

Outreach, and promoting sciences to underrepresented groups will help improve
the issue of diversity with the next generation of physicists.
At the point that students reach the college level, these norms are
well established and the best way for a faculty member to address diversity is through
flexibility.  We need to understand that underrepresented groups may not have the
same level of preparation (given differing socioeconomic conditions) or
confidence (given the discouragement from cultural norms) and we need the
flexibility to accept these discrepancies.  This does not mean that we
lower the standards, but rather that it is the obligation of the
instructor to provide individual attention to struggling students with the
understanding that these issues can be possibly influenced by inequality.

Diversity is a problem that is easy to simplify but difficult to influence. We
need to develop productive methods to inspire the next generation of physicists
in a way that is directed towards underrepresented groups. As a faculty member,
I would work to expand the outreach program in a targeted way and
have the flexibility to ensure that all students have a chance to succeed.



\end{document}
