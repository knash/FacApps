\documentclass[12pt]{article}
%\usepackage[
%backend=bibtex,
%citestyle=ieee
%]{biblatex}
%\addbibresource{RI2020.bib}
%\renewbibmacro{in:}{}
%% Define some new colors
\usepackage{xcolor}

\usepackage{scrextend}


%% Page and text formatting
\usepackage[left=1.0in, right=1.0in, top=1.0in, bottom=1.0in]{geometry} % margins
\usepackage{setspace}
\singlespacing % No more than 6 lines of text per inch
\usepackage{amsmath, amsfonts}
\usepackage[T1]{fontenc}
\usepackage{times}
\usepackage{graphics}
\usepackage{makecell}
\usepackage{tabularx}
\usepackage{indentfirst}

%% Set up hyperlinks
\usepackage[linktoc=page]{hyperref}
\hypersetup{
  colorlinks = true,
  urlcolor = blue,
  citecolor = blue,
  linkcolor = blue,
}

%% Reduce whitespace around lists (globally)
\usepackage{enumitem}
\setlist[itemize]{nosep,left=0pt}
\setlist[enumerate]{nosep}
\setlist[description]{nosep}

%% Formatting for sections

\usepackage{sectsty}
\sectionfont{\fontsize{22}{25}\selectfont}
\subsectionfont{\fontsize{15}{25}\selectfont}




\begin{document}

\fontsize{10}{12}\selectfont

\linespread{1.3}
\fontfamily{ptm}\selectfont

\vspace{2mm}

\vspace{1mm}

\noindent{\fontsize{15}{17}\selectfont Kevin Nash \\ }

\noindent\begin{tabular*}{\textwidth}{@{\extracolsep{\fill}}l l}
Research Associate & Email: \href{mailto:knash@physics.rutgers.edu}{knash@physics.rutgers.edu} \\
Rutgers University & Phone: (434) 760-1424\\
Department of Physics \& Astronomy \\
136 Frelinghuysen Rd\\
Piscataway, NJ 08854\\
\hline
\\
\\
\end{tabular*}




\section*{Teaching Statement}


As a postdoctoral researcher, my teaching experience largely revolves around
teaching-assistant work at Johns Hopkins University during my graduate studies.
This included directing introductory physics labs and running problem sessions
for the physics major oriented introductory classes as well as grading homework
assignments and exams. However, I have always considered helping new students
through their first introductions to the classwork and inspiring promising young
physicists in basic research opportunities the most rewarding aspects of the
field. Physics tends to be one of the more intimidating subjects for new students
but also one of the most closely associated with real world success. This is
because the skills learned in the classroom revolve around analytical problem
solving. Although the problems in introductory classes are typically contrived
and not directly applicable, the problem solving skills are of paramount
importance to any career. The challenge of a good introductory teacher is to
impress this upon the students.

The hallmark of a successful teaching style is student engagement. I have found
that much of the problem solving skill development occurs during group oriented
practice sessions. This requires the students to be engaged and allows them to
address confusion of the material among peers in an informal and unintimidating
environment, which in a unique advantage over an individual oriented session. I
have knowledge of how to successfully incorporate this type of learning through
my graduate teaching experience. During my work running problem solving sessions,
we have attempted a multiple teaching formats, but what I have found to promote
student engagement and interest was a cooperative problem solving system where a
rotating group “transcriber” would write the problem solution using a group
whiteboard while the other members would hypothesize potential solutions. Then,
a successful group could then display the solution to the class. This introduces
each student to methods of physical intuition and inherently requires a high
level of participation.

One of the most important times in a young high energy physicists career is the
first introduction to experimental research. This is a tumultuous time because
is poses a very different type of problem than is encountered in class. There are
many ways to make this transition, but it is essential to have a dedicated
advisor to teach skills in addressing the open ended and nebulous problems
encountered in physics research. A good introduction to research can usually be
accomplished by a basic detector characterization. This allows the student to
design the full apparatus of scientific research in a manageable scope. My
introduction to research at James Madison University was the testing of PMTs for
Jefferson Lab, specifically the measurement of dark current and gain variance
of different modules. This specifically offered an opportunity to fully characterize
a physics problem from the physical design of the testing apparatus to the data
analysis which serves additionally as a real world programming introduction. The
other possibility is a real world physics analysis, which is now a real option
for a young researcher. However, this also requires a more careful oversight to
make sure the student is able to develop the a diverse skill set due to the safety
net of legacy software packages can lead the student to not fully appreciate each
step in the analysis chain. During my postdoctoral research career, I have had the
opportunity to guide undergraduates through their very first CMS analyses, which
provides a head start on the programming and data analysis skills essential to
success in a graduate career.

A current challenge for the field is how to cultivate diversity. A diverse physics
community is objectively beneficial, but it is a pervasive problem that does not
seem to have an novel solution. The true solution lies only at a societal level
and can be influenced insofar as we can guide the long arc of cultural norms in
the direction of equality. However, we can incrementally improve the situation by
inspiring underrepresented young people. In my graduate career I participated in
outreach programs designed to inspire the next generation of physicists. Part of
this effort was to deliver a basic overview of LHC high energy physics research to
Baltimore-area high schools and participate in physics demonstrations that can seed
inspiration for the sciences. Additionally, these outreach experiences have given
me the experience necessary to break down complicated topics into a simple and
intuitive description.

An essential part of a physics course plan is malleability. An effective learning
strategy is something that if formed over years of experience and only if the
instructor can correlate teaching strategy with subject comprehension. Therefore,
I plan to constantly update my teaching strategy with class feedback and incorporating
techniques from professional educators. Additionally, I plan to encourage students
to seek individual help from both myself and more senior students and to actively
participate in class discussions or ask questions when a topic is confusing.
There exists an apprehension in asking for help, but in physics it can be unavoidable,
so therefore I will strive to encourage an environment that is free of this stigma.

\end{document}
